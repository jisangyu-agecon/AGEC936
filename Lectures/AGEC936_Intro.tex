% Options for packages loaded elsewhere
\PassOptionsToPackage{unicode}{hyperref}
\PassOptionsToPackage{hyphens}{url}
%
\documentclass[
  ignorenonframetext,
]{beamer}
\usepackage{pgfpages}
\setbeamertemplate{caption}[numbered]
\setbeamertemplate{caption label separator}{: }
\setbeamercolor{caption name}{fg=normal text.fg}
\beamertemplatenavigationsymbolsempty
% Prevent slide breaks in the middle of a paragraph
\widowpenalties 1 10000
\raggedbottom
\setbeamertemplate{part page}{
  \centering
  \begin{beamercolorbox}[sep=16pt,center]{part title}
    \usebeamerfont{part title}\insertpart\par
  \end{beamercolorbox}
}
\setbeamertemplate{section page}{
  \centering
  \begin{beamercolorbox}[sep=12pt,center]{part title}
    \usebeamerfont{section title}\insertsection\par
  \end{beamercolorbox}
}
\setbeamertemplate{subsection page}{
  \centering
  \begin{beamercolorbox}[sep=8pt,center]{part title}
    \usebeamerfont{subsection title}\insertsubsection\par
  \end{beamercolorbox}
}
\AtBeginPart{
  \frame{\partpage}
}
\AtBeginSection{
  \ifbibliography
  \else
    \frame{\sectionpage}
  \fi
}
\AtBeginSubsection{
  \frame{\subsectionpage}
}
\usepackage{amsmath,amssymb}
\usepackage{lmodern}
\usepackage{iftex}
\ifPDFTeX
  \usepackage[T1]{fontenc}
  \usepackage[utf8]{inputenc}
  \usepackage{textcomp} % provide euro and other symbols
\else % if luatex or xetex
  \usepackage{unicode-math}
  \defaultfontfeatures{Scale=MatchLowercase}
  \defaultfontfeatures[\rmfamily]{Ligatures=TeX,Scale=1}
\fi
% Use upquote if available, for straight quotes in verbatim environments
\IfFileExists{upquote.sty}{\usepackage{upquote}}{}
\IfFileExists{microtype.sty}{% use microtype if available
  \usepackage[]{microtype}
  \UseMicrotypeSet[protrusion]{basicmath} % disable protrusion for tt fonts
}{}
\makeatletter
\@ifundefined{KOMAClassName}{% if non-KOMA class
  \IfFileExists{parskip.sty}{%
    \usepackage{parskip}
  }{% else
    \setlength{\parindent}{0pt}
    \setlength{\parskip}{6pt plus 2pt minus 1pt}}
}{% if KOMA class
  \KOMAoptions{parskip=half}}
\makeatother
\usepackage{xcolor}
\newif\ifbibliography
\setlength{\emergencystretch}{3em} % prevent overfull lines
\providecommand{\tightlist}{%
  \setlength{\itemsep}{0pt}\setlength{\parskip}{0pt}}
\setcounter{secnumdepth}{-\maxdimen} % remove section numbering
\newlength{\cslhangindent}
\setlength{\cslhangindent}{1.5em}
\newlength{\csllabelwidth}
\setlength{\csllabelwidth}{3em}
\newlength{\cslentryspacingunit} % times entry-spacing
\setlength{\cslentryspacingunit}{\parskip}
\newenvironment{CSLReferences}[2] % #1 hanging-ident, #2 entry spacing
 {% don't indent paragraphs
  \setlength{\parindent}{0pt}
  % turn on hanging indent if param 1 is 1
  \ifodd #1
  \let\oldpar\par
  \def\par{\hangindent=\cslhangindent\oldpar}
  \fi
  % set entry spacing
  \setlength{\parskip}{#2\cslentryspacingunit}
 }%
 {}
\usepackage{calc}
\newcommand{\CSLBlock}[1]{#1\hfill\break}
\newcommand{\CSLLeftMargin}[1]{\parbox[t]{\csllabelwidth}{#1}}
\newcommand{\CSLRightInline}[1]{\parbox[t]{\linewidth - \csllabelwidth}{#1}\break}
\newcommand{\CSLIndent}[1]{\hspace{\cslhangindent}#1}
\ifLuaTeX
  \usepackage{selnolig}  % disable illegal ligatures
\fi
\IfFileExists{bookmark.sty}{\usepackage{bookmark}}{\usepackage{hyperref}}
\IfFileExists{xurl.sty}{\usepackage{xurl}}{} % add URL line breaks if available
\urlstyle{same} % disable monospaced font for URLs
\hypersetup{
  pdftitle={AGEC 936 - Module 1},
  pdfauthor={Jisang Yu   Kansas State University},
  hidelinks,
  pdfcreator={LaTeX via pandoc}}

\title{AGEC 936 - Module 1}
\subtitle{Introduction of Causal Inference}
\author{Jisang Yu Kansas State University}
\date{}

\begin{document}
\frame{\titlepage}

\begin{frame}{Introduction: Key materials}
\protect\hypertarget{introduction-key-materials}{}
\begin{itemize}
\item
  Slides
\item
  Readings
\item
  Useful references (no need to purchase, The Mixtape and The Effect are
  freely available online)

  \begin{itemize}
  \tightlist
  \item
    Cunningham, S., 2021, Causal Inference: The Mixtape,
    \href{https://mixtape.scunning.com/}{link}
  \item
    Hungington-Klein, N., 2022, The Effect: An Introduction to Research
    Design and Causality, \href{https://theeffectbook.net/}{link}
  \item
    Cameron, A.C. and Trivedi, P.K., 2005. Microeconometrics: Methods
    and Applications.\\
  \item
    Angrist, J.D. and Pischke, J.S., 2009. Mostly harmless econometrics:
    An empiricist's companion. Princeton university press.
  \end{itemize}
\end{itemize}
\end{frame}

\begin{frame}{Introduction: Assignments}
\protect\hypertarget{introduction-assignments}{}
\begin{itemize}
\item
  Problem set (most likely one)
\item
  Referee report (5 page max) and Presentation (Group assignment):
  Choose an empirical paper that utilizes a causal analysis. Choose from
  major economics or agricultural economics journals or NBER working
  papers (limit the time period to post-2010). Please submit the
  citation of the paper of your choice by \textbf{30\(^{th}\) of
  January}. Your referee report is due \textbf{17\(^{th}\) of February}.
  Your referee report should contain

  \begin{itemize}
  \tightlist
  \item
    A brief summary of the paper (0.5 page)
  \item
    Describe the main empirical specification and the identification
    strategy of the paper (1 -- 1.5 page)
  \item
    Your criticism on the identification strategy (1 page)
  \item
    Discuss possible alternative approaches (1 page)
  \item
    Conclude with the lessons you learned (1 page)
  \end{itemize}
\end{itemize}

Presentations will be held on \textbf{15th} and \textbf{17th} of
\textbf{February} (15 minutes each).
\end{frame}

\begin{frame}{Introduction: What we will cover}
\protect\hypertarget{introduction-what-we-will-cover}{}
\begin{enumerate}
\item
  Understanding causal inference
\item
  Fixed effects and Difference-in-Differences
\item
  Synthetic control
\item
  Panel IV/GMM
\item
  Shift-share design
\item
  \emph{Any other causal inference-related topics you are interested in?
  (Let me know by coming Friday)}
\end{enumerate}
\end{frame}

\begin{frame}{Introduction: My goals here are\ldots{}}
\protect\hypertarget{introduction-my-goals-here-are}{}
\begin{itemize}
\item
  To bring various (and mostly recent) empirical issues related to
  causal inference to your attention
\item
  To help you become a critical consumer and a careful producer of
  empirical studies
\item
  Not necessarily ``teach'' you new methods -- I will provide various
  frameworks for carefully thinking about these empirical issues and
  point you to the relevant literature.
\end{itemize}

\emph{These being said, you may will leave this module with more
questions than answers after this module\ldots{}}
\end{frame}

\begin{frame}{Causal inference: an introduction}
\protect\hypertarget{causal-inference-an-introduction}{}
Following the framework of Angrist and Pischke (2008), suppose we have a
binary treatment variable:

\[
D_i=\{0,\;1\} 
\] and the outcome of interest is denoted as \(Y_i\) (For example, in
the book, health status of individual \(i\) is \(Y_i\)). And the
potential outcomes for individual \(i\) are denoted as

\[
Y_{0i}\;if\;D_i=0, \\
Y_{1i}\;if\;D_i=1. \\
\]

The causal effect of the treatment \(D\) is the difference between
\(Y_{0i}\) and \(Y_{1i}\).

\emph{Can we observe both \(Y_{0i}\) and \(Y_{1i}\) (Rubin (1974))?}
\end{frame}

\begin{frame}{What are we observing in practice}
\protect\hypertarget{what-are-we-observing-in-practice}{}
We normally observe \[
Y_i=Y_{0i}+(Y_{1i}-Y_{0i})D_{i}.
\]

The average treatment (or causal) effect (ATE) is defined as
\begin{equation}
E(Y_{1i}-Y_{0i})=E(Y_{1i})-E(Y_{0i}).
\end{equation}

What we observe is \begin{equation}
E(Y_{i}|D_i=1)-E(Y_{i}|D_i=0) (\#eq:OD)
\end{equation}

\emph{Is the observed difference ATE?}

\emph{When does the observed difference become ATE?}
\end{frame}

\begin{frame}{Selection Bias}
\protect\hypertarget{selection-bias}{}
\emph{What happens if \(Y_{1i}\), \(Y_{0i}\), and \(D_i\) are not
independent?}

\begin{equation}
E(Y_{i}|D_i=1)-E(Y_{i}|D_i=0) \\
=E(Y_{0i}+(Y_{1i}-Y_{0i})|D_i=1)-E(Y_{0i}|D_i=0) \\
=\underbrace{(E(Y_{1i}|D_i=1)-E(Y_{0i}|D_i=1))}_\text{Average Treatment Effect on Treated (ATT)}+\underbrace{(E(Y_{0i}|D_i=1)-E(Y_{0i}|D_i=0))}_\text{Selection Bias}.
\end{equation}

\emph{How do we interpret ``ATT''?}
\end{frame}

\begin{frame}{Regression representation}
\protect\hypertarget{regression-representation}{}
We can rewrite \[
Y_i=Y_{0i}+(Y_{1i}-Y_{0i})D_{i}.
\] as the following regression equation:

\[
Y_i=\alpha+\rho D_{i}+\varepsilon_i
\]

and we know that OLS estimates of \(\rho\) is

\[
\hat{\rho}=\frac{\sum (D_i - \bar{D})(Y_i-\bar{Y})}{\sum (D_i - \bar{D})^2} 
\]

\[
E(\hat{\rho})=E\left(\frac{\sum (D_i - \bar{D})\left(\rho (D_i-\bar{D})+\varepsilon_i \right)}{\sum (D_i - \bar{D})^2} \right)\\
=E\left(\frac{\sum (D_i - \bar{D})\left(\rho (D_i-\bar{D})\right)}{\sum (D_i - \bar{D})^2}\right)+E\left(\frac{\sum \left(\varepsilon_i (D_i-\bar{D})\right)}{\sum (D_i - \bar{D})^2} \right)\\
=\rho+E\left(\frac{\sum \left(\varepsilon_i (D_i-\bar{D})\right)}{\sum (D_i - \bar{D})^2} \right)\\
=\rho+\frac{Cov(\varepsilon_i,\;D_i)}{Var(D_i)}
\] \emph{What is the selection bias term here?}
\end{frame}

\begin{frame}{Conditional Independence (Rosenbaum and Rubin (1983))}
\protect\hypertarget{conditional-independence-rosenbaum1983central}{}
Let us denote an observable characteristic of individual \(i\) as
\(X_i\). Now, what we observe is \[
E(Y_i|D_i=1,\;X_i)-E(Y_i|D_i=0,\;X_i)= \\
E(Y_{1i}|D_i=1,\;X_i)-E(Y_{0i}|D_i=0,\;X_i).
\]

Conditional independence assumption, i.e.~conditional on \(X_i\),
\((Y_{0i},\;Y_{1i})\), and \(D_i\) are independent, implies \[
E(Y_{0i}|D_i=0,\;X_i)=E(Y_{0i}|D_i=1,\;X_i)=E(Y_{0i}|X_i), \\
E(Y_{1i}|D_i=1,\;X_i)=E(Y_{1i}|D_i=0,\;X_i)=E(Y_{1i}|X_i). \\
\]

Because of the conditional independence, the observed difference becomes
\[
E(Y_{1i}|D_i=1,\;X_i)-E(Y_{0i}|D_i=0,\;X_i)= \\
E(Y_{1i}|D_i=1,\;X_i)-E(Y_{0i}|D_i=1,\;X_i)= \\
E(Y_{1i}-Y_{0i}|D_i=1,\;X_i).
\]

By the law of iterated expectation, we have \[
E(E(Y_{1i}-Y_{0i}|D_i=1,\;X_i))=E(Y_{1i}-Y_{0i}|D_i=1).
\]
\end{frame}

\begin{frame}{Matching (Angrist (1998))}
\protect\hypertarget{matching-angrist1998}{}
For the simplicity, let us assume that \(X_i\) is also binary (also,
assume that there is at least one observation for each pair of
\((D_i,\;X_i)\)). Using the law of iterated expectation, the ATT can be
represented as

\[
E(Y_{1i}-Y_{0i}|D_i=1)=\\
E(E(Y_{1i}-Y_{0i}|D_i=1,\;X_i))=\\
E(Y_{1i}-Y_{0i}|D_i=1,\;X_i=0) \times P(X_i=0|D_i=1)+ \\
E(Y_{1i}-Y_{0i}|D_i=1,\;X_i=1) \times P(X_i=1|D_i=1).
\]

In Angrist (1998), conditional treatment effects (on the treated) are
estimated by taking the differences between the earnings of veterans and
the earnings of nonveterans conditional on \(X_i\). The weights
(conditional probabilities) are the population distribution of \(X_i\)s
among veterans.
\end{frame}

\begin{frame}{Regression}
\protect\hypertarget{regression}{}
We can write down \(Y_i\) as \[
Y_i=\beta_0+\beta_1 X_i+ \beta_2 D_i+\epsilon_i.
\]

Following Angrist (1998), the regression coefficient \(\beta_2\), is \[
\beta_2=\frac{E(Y_{1i}-Y_{0i}|D_i=1,\;X_i=0) \times P(D_i=1|X_i=0)(1-P(D_i=1|X_i=0))P(X_i=0)}{E(P(D_i=1|X_i)(1-P(D_i=1|X_i)))}+\\
\frac{E(Y_{1i}-Y_{0i}|D_i=1,\;X_i=1) \times P(D_i=1|X_i=1)(1-P(D_i=1|X_i=1))P(X_i=1)}{E(P(D_i=1|X_i)(1-P(D_i=1|X_i)))}.
\]
\end{frame}

\begin{frame}{Next slides}
\protect\hypertarget{next-slides}{}
\href{https://jisangyu-agecon.github.io/AGEC936/Lectures/AGEC936_DD_FE_Synth.html}{DID
and TWFE}
\end{frame}

\begin{frame}{References}
\protect\hypertarget{references}{}
\hypertarget{refs}{}
\begin{CSLReferences}{1}{0}
\leavevmode\vadjust pre{\hypertarget{ref-angrist1998}{}}%
Angrist, Joshua D. 1998. {``Estimating the Labor Market Impact of
Voluntary Military Service Using Social Security Data on Military
Applicants.''} \emph{Econometrica}, 249--88.

\leavevmode\vadjust pre{\hypertarget{ref-angrist2008mostly}{}}%
Angrist, Joshua D, and Jörn-Steffen Pischke. 2008. \emph{Mostly Harmless
Econometrics: An Empiricist's Companion}. Princeton university press.

\leavevmode\vadjust pre{\hypertarget{ref-rosenbaum1983central}{}}%
Rosenbaum, Paul R, and Donald B Rubin. 1983. {``The Central Role of the
Propensity Score in Observational Studies for Causal Effects.''}
\emph{Biometrika}, 41--55.

\leavevmode\vadjust pre{\hypertarget{ref-rubin1974estimating}{}}%
Rubin, Donald B. 1974. {``Estimating Causal Effects of Treatments in
Randomized and Nonrandomized Studies.''} \emph{Journal of Educational
Psychology} 66 (5): 688.

\end{CSLReferences}
\end{frame}

\end{document}
